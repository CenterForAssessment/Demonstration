\documentclass[]{tufte-book}

%  AVI - remove the "double-sided" effect (blank page between chapters) - https://tex.stackexchange.com/questions/111580/removing-an-unwanted-page-between-two-chapters
\let\cleardoublepage\clearpage
%

% ams
\usepackage{amssymb,amsmath}

\usepackage{ifxetex,ifluatex}
\usepackage{fixltx2e} % provides \textsubscript
\ifnum 0\ifxetex 1\fi\ifluatex 1\fi=0 % if pdftex
  \usepackage[T1]{fontenc}
  \usepackage[utf8]{inputenc}
\else % if luatex or xelatex
  \makeatletter
  \@ifpackageloaded{fontspec}{}{\usepackage{fontspec}}
  \makeatother
  \defaultfontfeatures{Ligatures=TeX,Scale=MatchLowercase}
  \makeatletter
  \@ifpackageloaded{soul}{
     \renewcommand\allcapsspacing[1]{{\addfontfeature{LetterSpace=15}#1}}
     \renewcommand\smallcapsspacing[1]{{\addfontfeature{LetterSpace=10}#1}}
   }{}
  \makeatother

\fi

% graphix
\usepackage{graphicx}
\setkeys{Gin}{width=\linewidth,totalheight=\textheight,keepaspectratio}

% booktabs
\usepackage{booktabs}

% url
\usepackage{url}

% hyperref
\usepackage{hyperref}

% units.
\usepackage{units}


\setcounter{secnumdepth}{-1}

% citations
\usepackage{natbib}
\bibliographystyle{plainnat}

% pandoc syntax highlighting

% longtable

% multiplecol
\usepackage{multicol}

% strikeout
\usepackage[normalem]{ulem}

% morefloats
\usepackage{morefloats}


% tightlist macro required by pandoc >= 1.14
\providecommand{\tightlist}{%
  \setlength{\itemsep}{0pt}\setlength{\parskip}{0pt}}

\usepackage{booktabs}
\usepackage[svgnames]{xcolor}
\usepackage{geometry}
\hyphenation{thatshouldnot}

\usepackage{makeidx}
\makeindex

\newcommand{\subtitle}[1]{
  \gdef\@subtitle{#1}
}

\renewcommand{\maketitlepage}{%
  \cleardoublepage%
  {%
  \sffamily%
  \pagecolor{Cornsilk}
  \newgeometry{left=2.0cm, right=2cm, top=1cm, bottom=0.5cm}
  \vspace*{1in}%
  \fontsize{50}{54}\selectfont\par\noindent{S\textcolor{SlateGrey}{tudent}\\\vspace{-0.05in}\noindent{G\textcolor{SlateGrey}{rowth}}\\\vspace{0.15in}\noindent{P\textcolor{SlateGrey}{ercentiles}}}%
  \vspace{0.5in}%
  \fontsize{24}{26}\selectfont\par\noindent\textcolor{SlateGrey}{Theory and Calculation}%
  \vspace{0.75in}%
  \fontsize{14}{16}\selectfont\par\noindent\textcolor{SlateGrey}{\thanklessauthor}%
  \vspace{3.5in}%
  \begin{center}
  \includegraphics[width=0.7in]{images/CFA_Logo.png}
  \fontsize{14}{16}\selectfont\par\noindent\textcolor{SlateGrey}{National Center for the Improvement of Educational Assessment}%
  \end{center}
  }
  \thispagestyle{empty}%
  \restoregeometry%
  \pagecolor{White}
  \pagebreak
}

% From "user11232" at:
% http://tex.stackexchange.com/questions/167526/how-to-write-dedication-properly

\newenvironment{dedication}
  {%\clearpage           % we want a new page          %% I commented this
   \thispagestyle{empty}% no header and footer
   \vspace*{\stretch{1}}% some space at the top
   \itshape             % the text is in italics
   \raggedleft          % flush to the right margin
  }
  {\par % end the paragraph
   \vspace{\stretch{3}} % space at bottom is three times that at the top
   \clearpage           % finish off the page
  }

% title / author / date
\title{Student Growth Percentiles}
\author{Damian W Betebenner \& Adam R VanIwaarden}
\date{October 12th, 2017}

\begin{document}

\maketitle


%  No abstract for packagePages::tufte_book
% \begin{abstract}
% \noindent This document provides an introduction to Student Growth Percentiles
(SGP)
% \end{abstract}

\begin{dedication}
For \c{C}i\u{g}dem \& \c{S}\^{i}lan, all of my love.
\end{dedication}

{
\setcounter{tocdepth}{1}
\tableofcontents
}

\chapter{Preface}\label{preface}

The following provides a brief introduction to generalized additive
models and some thoughts on getting started within the R environment. It
doesn't assume much more than a basic exposure to regression, and maybe
a general idea of R though not necessarily any particular expertise. The
presentation is of a very applied nature, and such that the topics build
upon the familiar and generalize to the less so, with the hope that one
can bring the concepts they are comfortable with to the new material.
The audience in mind is a researcher with typical applied science
training.

As this document is more conceptual, a basic familiarity with R is all
that is needed to follow the code, though there is much to be gained
from simple web browsing on R if one needs it. And while it wasn't the
intention starting out, this document could be seen as a vignette for
the {mgcv} package, which is highly recommended.

This section provides basic details about the calculation of student
growth percentiles from Georgia state assessment data using the
\href{http://www.r-project.org/}{\texttt{R} Software Environment}
\citep{Rsoftware} in conjunction with the
\href{https://github.com/CenterForAssessment/SGP}{\texttt{SGP} package}
\citep{sgp2017}. For a more in-depth discussion of SGP calculation, see
Betebenner \citeyearpar{Betebenner:2009}, and see Shang, VanIwaarden and
Betebenner \citeyearpar{ShangVanIBet:2015} and Appendix B of this report
for further information on the SIMEX measurement error correction
methodology.

This document was created with \href{http://rstudio.org/}{Rstudio} and
rmarkdown. {Last modified 2017-11-01. Original draft August, 2012.}

\cite{Betebenner:2009}

Color guide:

\begin{itemize}
\tightlist
\item
  {important term}
\item
  \href{}{link}
\item
  {package}
\item
  {function}
\item
  {object or class}
\end{itemize}

R Info: \textbf{R version 3.4.2 (2017-09-28) Short Summer}

\chapter{Conclusion}\label{conclusion}

Generalized additive models are a conceptually straightforward tool that
allows one to incorporate nonlinear predictor effects into their
otherwise linear models. In addition, they allow one to keep within the
linear and generalized linear modeling frameworks with which one is
already familiar, while providing new avenues of model exploration and
possibly improved results. As was demonstrated, it is easy enough with
just a modicum of familiarity to pull them off within the R environment,
and as such, it is hoped that this document provides the means to do so.

\renewcommand\bibname{References}
\bibliography{/Library/Frameworks/R.framework/Versions/3.4/Resources/library/Literasee/rmarkdown/content/bibliography/Literasee.bib}



\end{document}
